

Det finns fem tärningar t med värdern mellan 1 till 6

\begin{equation}
    \Omega=\{t_1,...,t_5:t_i\in\{1,...,6\}\}
\end{equation}

Ett sett av tärningskastningar kommer kallas för $T$

Reglerna går ut så här:
\begin{itemize}
    \item Första rundan kastar man alla fem tärningar.
    \item Sedan får man själv välja vilka tärningar som ska kastas. Man får två sådana försök på sig.
    \item Det finns då en hel lista av olika kategorier spelaren kan sätta in på resultaten. Ett exempel kanske är $d_1=2, d_2=2$, 
då kan man antingen välja att sätta in det i "par" kategorin eller i "två" kategorin.
\end{itemize}

Listan på alla kategorier k och deras accepterade konfigurationer:
\begin{itemize}
    \item $K_{N,n} \rightarrow n=t_i=t_{i+1}...=t_{i+N}$
    \item $K_{N} \rightarrow t_i=t_{i+1}...=t_{i+N}$
    \item $K_{lagrestege} \rightarrow t_0>t_1>...t_{N-1}$
    \item $K_{hogrestege} \rightarrow t_1>t_2>...t_{N}$
\end{itemize}

$K_3$ är ett exempel på "triss", d.v.s man får tre av samma figur. $K_{4,1}$ betyder att man har fått 4 stycken tärningar med 1.

Poängen man får för varje konfiguration förutom yatzee ($|K_{N,n}|=N)$ är 

\begin{equation}
    \sum_{t_i\in K}t_i 
\end{equation}





